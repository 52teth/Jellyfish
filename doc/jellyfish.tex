\documentclass[english]{article}
\usepackage[latin1]{inputenc}
\usepackage{babel}
\usepackage{verbatim}

%% do we have the `hyperref package?
\IfFileExists{hyperref.sty}{
   \usepackage[bookmarksopen,bookmarksnumbered]{hyperref}
}{}

%% do we have the `fancyhdr' package?
\IfFileExists{fancyhdr.sty}{
\usepackage[fancyhdr]{latex2man}
}{
%% do we have the `fancyheadings' package?
\IfFileExists{fancyheadings.sty}{
\usepackage[fancy]{latex2man}
}{
\usepackage[nofancy]{latex2man}
\message{no fancyhdr or fancyheadings package present, discard it}
}}

\newcommand{\ddash}[1]{-\,-#1}
\newcommand{\LOpt}[1]{\Opt{\ddash{#1}}}
\newcommand{\LoOpt}[1]{\oOpt{\ddash{#1}}}
\newcommand{\LOptArg}[2]{\OptArg{\ddash{#1}}{#2}}
\newcommand{\LoOptArg}[2]{\oOptArg{\ddash{#1}}{#2}}
\newcommand{\LoOptoArg}[2]{\oOptoArg{\ddash{#1}}{#2}}

\setVersion{1.0}
\setDate{2010/10/1}

\begin{document}

\begin{Name}{1}{jellyfish}{G. Marcais}{k-mer counter}{Jellyfish: A fast k-mer counter}

\Prog{Jellyfish} is a software to count $k$-mers in DNA sequences.

\end{Name}

\section{Synopsis}
\Prog{jellyfish count} \oOptArg{-o}{prefix} \oOptArg{-m}{merlength} \oOptArg{-t}{threads} \oOptArg{-s}{hashsize} \LoOpt{both-strands} \Arg{fasta} \oArg{fasta \Dots} \\
\Prog{jellyfish merge} \Arg{hash1} \Arg{hash2} \Dots \\
\Prog{jellyfish stats} \LoOpt{fasta} \oOpt{-c} \Arg{hash} \\
\Prog{jellyfish histo} \oOptArg{-h}{high} \oOptArg{-l}{low} \oOptArg{-i}{increment} \Arg{hash} \\
\Prog{jellyfish query} \Arg{hash}

\section{Description}

\Prog{Jellyfish} is a $k$-mer counter based on a multi-threaded hash
table implementation.

To count $k$-mers, use a command like:

\begin{verbatim}
jellyfish count -m 22 -o output -c 3 -s 10000000 -t 32 input.fasta
\end{verbatim}

This will count the the 22-mers in species.fasta with 32 threads. The
counter field in the hash uses only 3 bits and the hash has at least
10 million entries. Let the size of the table be $s=2^l$ and the max
reprobe value is less than $2^r$, then the memory usage per entry in the hash is (in bits, not bytes) $2k-l+r+1$.

To save space, the hash table supports variable length counter, i.e. a
$k$-mer occurring only a few times will use a small counter, a $k$-mer
occurring many times will used multiple entries in the hash. The
\Opt{-c} specify the length of the small counter. The tradeoff is: a
low value will save space per entry in the hash but will increase the
number of entries used, hence maybe requiring a larger hash. In
practice, use a value for \Opt{-c} so that most of you $k$-mers
require only 1 entry. For example, to count $k$-mers in a genome,
where most of the sequence is unique, use \OptArg{-c}{1} or
\OptArg{-c}{2}. For sequencing reads, use a value for
\Opt{-c} large enough to counts up to twice the coverage.

When the orientation of the sequences in the input fasta file is not
known, e.g. in sequencing reads, using \LOpt{both-strands} makes the most sense.

\section{Options}

\subsection{count}

Count $k$-mers in one or many fasta file(s). There is no restriction
in the size of the fasta file, the number of sequences or the size of
the sequences in the fasta files. On the other hand, they must be
files on and not pipes, as the files are memory mapped into memory.

\begin{description}
\item[\Opt{-o},\LOptArg{output=}{prefix}] Output file prefix. Results will be store
  in files with the format \File{prefix\_\#}, where \# is a number
  starting at 0. More than one file will be written if all the
  $k$-mers could not be counted in the given size.
\item[\Opt{-m},\LOptArg{mer-len=}{merlength}] Length of mer to
  count. I.e. value of $k$ in $k$-mer.
\item[\Opt{-t},\LOptArg{threads=}{NB}] Number of threads.
\item[\Opt{-s},\LOptArg{size=}{hashsize}] Size of hash table. This
  will be rounded up to the next power of two.
\item[\LOpt{both-strands}] Collapse counters for a $k$-mer and its
  reverse complement. I.e., when \Prog{jellyfish} encounters a $k$-mer
  $m$, it checks which of $m$ or the reverse complement of $m$ comes
  first in lexicographic order (call it the canonical representation)
  and increments the counter for this canonical representation.
\item[\Opt{-p},\LOptArg{reprobes=}{NB}] Maximum reprobe
  value. This determine the usage of the hash table (i.e. \% of
  entries used in hash) before being deemed full and written to disk.
\item[\LOptArg{timing=}{file}] Write detailed timing information to
  \File{file}.
\item[\LOpt{no-write}] Do not write result file. Used for timing only.
\item[\LOptArg{out-counter-len=}{LEN}] Length of the counter field in
  the output (in bytes). The value of the counter for any $k$-mer is
  capped at the maximum value that can be encoded in this number of
  bytes.
\end{description}

\subsection{stats}

Display statistics or dump full content of hash table in an easily
parsable text format.

\begin{description}
\item[\Opt{-c},\LOpt{column}] Print $k$-mers counts in column format: sequence count.
\item[\Opt{-f},\LOpt{fasta}] Print $k$-mers counts in fasta format. The header is the count.
\item[\Opt{-r},\LOpt{recompute}] Recompute statistics from the hash
  table instead of the statistics in the header.
\end{description}

By default, it displays the statistics in the header of the file. These are:

\begin{description}
\item[Unique:] Number of $k$-mers which occur only once.
\item[Distinct:] Number of $k$-mers, not counting multiplicity.
\item[Total:] Number of $k$-mers including multiplicity.
\item[Max\_count:] Maximum number of occurrence of a $k$-mer.
\end{description}

\subsection{histo}

Create an histogram with the number of $k$-mers having a given count. In bucket $i$ are tallied the $k$-mers which have a count $c$ satisfying $low+i*inc<=c<low+(i+1)*inc$. Buckets in the output are labeled by the low end point ($low+i*inc$).

The last bucket in the output behaves as a catchall: it tallies all
$k$-mers with a count greater or equal to the low end point of this
bucket.

\begin{description}
\item[\Opt{-h},\LOptArg{high=}{HIGH}] High count bucket value.
\item[\Opt{-i},\LOptArg{increment=}{INC}] Increment for bucket value.
\item[\Opt{-l},\LOptArg{low=}{LOW}] Low count bucket value.
\end{description}

\subsection{query}

Query a database created with \Prog{jellyfish count}. It reads
$k$-mers from the standard input and write the counts on the standard
output. For example:

\begin{verbatim}
$ echo "AAAAA ACGTA" | jellyfish query database
AAAAA 12
ACGTA 3
\end{verbatim}

\begin{description}
\item[\LOpt{both-strands}] Report the count for the canonical version of
  the $k$-mers read from standard input.
\end{description}

\section{Version}

Version: \Version\ of \Date

\section{Bugs}

\begin{itemize}
\item \Prog{jellyfish merge} has not been parallelized and is very
  slow.
\end{itemize}

\section{Copyright \& License}
\begin{description}
\item[Copyright] \copyright\ 2010, Guillaume Marcais \Email{guillaume@marcais.net} and Carl Kingsford \Email{carlk@umiacs.umd.edu}.

\item[License] This program is free software: you can redistribute it
  and/or modify it under the terms of the GNU General Public License
  as published by the Free Software Foundation, either version 3 of
  the License, or (at your option) any later version. \\
  This program is distributed in the hope that it will be useful, but
  WITHOUT ANY WARRANTY; without even the implied warranty of
  MERCHANTABILITY or FITNESS FOR A PARTICULAR PURPOSE.  See the GNU
  General Public License for more details. \\
  You should have received a copy of the GNU General Public License
  along with this program.  If not, see
  <\URL{http://www.gnu.org/licenses/}>.
\end{description}

\section{Authors}
\noindent
Guillaume Marcais \\
University of Maryland \\
\Email{gmarcais@umd.edu}

Carl Kingsford \\
University of Maryland \\
\Email{carlk@umiacs.umd.edu}

\LatexManEnd
\end{document}
